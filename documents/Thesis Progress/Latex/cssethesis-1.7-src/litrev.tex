\chapter{Literature Review}

In this chapter I will demonstrate some of the extended citation
capabilities provided by the {\sf natbib} package \cite{Dal1999}. It
replaces the standard \LaTeX\ \verb+\cite+ command with two basic forms of
citation command: \verb+\citep+ and \verb+\citet+, as well as providing
several other very useful ones.

The \verb+\citep+ command is best used when placing a citation at the end
of sentence or phrase (as above).  In the {\sf natbib} documentation, this
is referred to as a \emph{parenthetical citation}.%
\footnote{For ease of conversion from exisiting \LaTeX\ documents, you
might find it useful to place
\texttt{$\backslash$renewcommand\{$\backslash$cite\}\{$\backslash$citep\}}
in the preamble of the document, since any existing standard
\texttt{$\backslash$cite} commands should almost certainly be treated as
\texttt{$\backslash$citep}.}

When you want to refer to the authors of a particular work, typically at
the start of a sentence, a parenthetical citation is not appropriate. This
is particularly so if you are using a numerical or symbolic citation style.
You should \emph{not} start a sentence with
\begin{quote}
[2] says that this is most certainly \ldots
\end{quote}
In such situations you really need to give the authors' names. The
\verb+\citet+ command produces \emph{textual citations}, which allows you
to produce things like:
\begin{quote}
\citet{Ade1983} describes a means by which textures may be
characterized \ldots another approach is given in \citet{DeV1998}.

\citet{AGR1996} note that humans have little or no difficulty in
perceiving shape, yet find it extremely difficult to \emph{describe} what
they perceive.
\end{quote}

Note that an abbreviated version of the authors' names has been produced in
the third example above.  It is often desirable to have the full list of
authors' names given when a work  is first cited, and an abbreviated list
thereafter. This can be achieved by passing the \texttt{longnamesfirst}
option to {\sf natbib} when the package is used. This will produce
an initial citation like:
\begin{quote}
\citet*{AGR1996} note that humans have little or no difficulty in
perceiving shape, yet find it extremely difficult to \emph{describe} what
they perceive.
\end{quote}

Both the \verb+\citep+ and \verb+\citet+ can take two optional arguments.
If just one is provided,  its text will appear as a ``post-note'' after the
citation details. If two arguments are provided, the first defines a
pre-note, and the second a post-note. Here is an example:
\begin{quote}
\verb+\citep[Ch.~3]{AaK1989}+ \ldots{} \citep[Ch.~3]{AaK1989}\\
\verb+\citep[see][Ch.~3]{AaK1989}+ \ldots{} \citep[see][Ch.~3]{AaK1989}\\
\end{quote}

These examples only scratch the surface of what the {\sf natbib} package
can do. To discover the full power of the package, see the documentation at
CTAN \citep{Dal1999}. You probably already have it on your system. Try
\verb+locate natbib.dvi+ at the command line.
